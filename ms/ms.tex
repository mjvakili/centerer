\documentclass[12pt, preprint]{aastex}
\usepackage{graphicx}	% For figures
\usepackage{natbib}	% For citep and citep
\usepackage{amsmath}	% for \iint
\usepackage{bbm}
\usepackage[breaklinks]{hyperref}	% for blackboard bold numbers
\usepackage{hyperref}
\usepackage{enumitem}
\hypersetup{colorlinks}
\usepackage{color}
\usepackage{morefloats}
\definecolor{darkred}{rgb}{0.5,0,0}
\definecolor{darkgreen}{rgb}{0,0.5,0}
\definecolor{darkblue}{rgb}{0,0,0.5}
\hypersetup{ colorlinks,
linkcolor=darkblue,
filecolor=darkgreen,
urlcolor=darkred,
citecolor=darkblue }

%----- typeset certain kinds of words
\newcommand{\latin}[1]{\textit{#1}}
\DeclareMathOperator*{\argmax}{arg\,max}
\newcommand{\beq}{\begin{equation}}
\newcommand{\eeq}{\end{equation}}
\newcommand{\eg}{\latin{e.g.}}
\newcommand{\etal}{et~al.}
\newcommand{\etc}{\latin{etc.}}
\newcommand{\ie}{\latin{i.e.}}

%----- math shih
\newcommand{\given}{\,|\,}

% these are already defined in aastex package
%----- typeset journals  
%\newcommand{\aj}{Astron.\,J.}
%\newcommand{\apj}{Astrophys.\,J.}
%\newcommand{\apjl}{Astrophys.\,J.\,Lett.}
%\newcommand{\apjs}{Astrophys.\,J.\,Supp.\,Ser.}
%\newcommand{\mnras}{Mon.\,Not.\,Roy.\,Ast.\,Soc.}
%\newcommand{\pasp}{Pubs.\,Astron.\,Soc.\,Pac.}
%\newcommand{\aap}{Astron.\,\&~Astrophys.}

\newcommand{\todo}[1]{{\em \textcolor{red}{ #1}}}

\begin{document}

\author{
  Mohammadjavad~Vakili\altaffilmark{1},
  David~W.~Hogg\altaffilmark{1,2,3}}
\altaffiltext{1}{Center for Cosmology and Particle Physics, Department of Phyics,
             New York University, 4 Washington Pl., room 424, New York, NY, 10003, USA}
\altaffiltext{2}{Max-Planck-Institut f\"ur Astronomie, K\"onigstuhl 17, D-69117 Heidelberg, Germany}
\altaffiltext{3}{Center for Data Science, New York University, 726 Broadway, 7th Floor, New York, NY 10003, USA}
\email{mjvakili@nyu.edu}

\title{Do fast stellar centroiding methods saturate the Cram\'{e}r-Rao lower bound?}

\begin{abstract}
One of the most demanding tasks in astronomical image processing---in terms of precision---is 
the centroiding of stars. Upcoming large surveys are going to take images of 
billions of point sources, including many faint stars, with short exposure times. 
Real-time estimation of the centroids of stars is crucial for real-time PSF estimation, 
and maximal precision is required for measurements of proper motion. 

The fundamental Cram\'{e}r-Rao lower bound sets a limit on the root-mean-squared-error 
achievable by optimal estimators. Information-preserving estimators are able to deliver 
root-mean-squared-errors lower than the Crame\'{e}r-Rao bound. In this work, we aim to compare 
the performance of various centroiding methods, in terms of saturating the bound, when they 
are applied to relatively low signal-to-noise ratio unsaturated stars assuming zero-mean 
constant Gaussian noise. In order to make this comparison, we present the root-mean-squared-errors of 
these estimators and their corrsponding Crame\'{e}r-Rao bound as a function of the signal-to-noise ratio 
and the full-width at half-maximum of faint stars. 

We discuss two general circumstances in centroiding of faint stars: (i) when we have a good estimate
of the PSF, (ii) when we do not know the PSF. In the case that we know the PSF, 
we show that a fast polynomial centroiding after smoothing the image by the PSF can be 
as information-reserving as the maximum-likelihood estimator with full PSF profile fitting. 
In the case that we do not know the PSF, we demonstrate that although polynomial centroiding 
is not as optimal as PSF profile fitting, it comes very close to saturating the Cram\'{e}r-Rao lower bound
in a wide range of conditions. We also show that the moment-based method of center-of-light 
never comes close to saturating the bound, and thus it does not deliver reliable estimates of centroids.    
\end{abstract}

\keywords{methods: statistical --- methods: data analysis --- techniques: image processing}

\section{Introduction}

Accuarate estimates of the centers of point sources, which are convolved with telescope point spread function (and atmospheric PSF in case of ground based telescopes), and also the pixel response function, are crucial to further steps of
astronomical image processing. For instance, proper measurement of the shapes of galaxies
requires interpolation of the PSF estimates from the positions of stars across the
image to the positions of galaxies. At the position of each star, the PSF is estimated by sub-pixel 
shifting of the star so that it is centered on its centroid. If the sub-pixel shifts are wrong, then 
the PSF estimates will be biased. Moreover, measurements of the parallaxes and the proper motions of stars
depends on how well we can measure their centroids. 

Ideally, we want a centroiding procedure that provides measurements as precise 
as possible, without putting a huge computational burden on the photometric pipeline.
Reducing the computational cost becomes even more important in large surveys,
where we want to estimate the centroids of thousands of point sources detected
on the telescope's focal plane, for various real-time applications.

In this paper, we study the optimality of various techniques for centroiding 
faint, unsaturated stars. Our requirement for optimality is saturation of the 
theoretically-set lower bound, known as the Cram\'{e}r-Rao lower bound by the 
centroiding methods considered in this study. We apply a number of 
centroiding methods to a large number of simulated faint stars, assuming uncorrelated 
Gaussian noise, with different signal-to-noise ratio and size realizations. 
The Cram\'{e}r-Rao lower bound has an inverse relation with the signal-to-noise-ratio 
of stars. In the context of astrometry, the Cram\'{e}r-Rao lower bound saturation for least-squares 
estimators has been tested in specific limits in which the centroiding bias is 
negligible (\citealt{lobos}). 

Saturating the Cram\'{e}r-Rao lower bound in estimating the centroids of stars however, is limited
by the lack of knowledge about the exact shape of the PSF, also presence of noise. 
There are many sources of noise such as the CCD readout noise, sky noise, errors resulting from 
\todo{incorrect} flatfield corrections, and photon noise from the astronomical object itself. 
In this study, we limit our investigation to the simulated images that contain non-overlapping faint sources 
that are sky-limitted. We assume that the generated images are not affected by flatfield corrections. 

However, we focus the scope of this investigation to sky-limited images for which the sky level has been 
subtracted. Furthermore, we assume that any instruement gain has been calibrated out, and that the 
simulated images are free of any contamination by cosmic rays, stray light from neighboring fields, or 
any other type of defect in real images. We expect these defects to move the centroiding errors further from the 
fundamental bound. We intend to investigate whether fast centroiding estimates can saturate the bound 
in a realistic range of low signal-to-noise ratio images that are sky-limited. 


%To date, a number of softwares packages have been designed for the purpose of extracting astronomical
%sources and making catalogs. One of these softwares is SExtractor (\citealt{sextractor}),
%whose centroiding method involves first finding the zeroth moment of the object
%as a first-order estimate, and then iteratively correcting the centroid by computing
%the zeroth order moment of the object weighted by a Gaussian window function,
%until the correction falls below a particular threshold value.
%The width of the Gaussian window function is set by the object's half-light radius.
%Other examples are DAOPHOT (\citealt{daophot}), and DOPHOT (\citealt{dophot})
%which both assume analytic models for the stellar PSF profiles with centroid
%coordinates being free parameters of these models.
%DAOPHOT (DOPHOT) finds the centroids by fitting a Gaussian (a truncated power series) to 
%the light profile of stars.

Given an analytic expression for the PSF model adopted in this study,
we derive an expression for the fundamental lower bound on the centroiding error as
a function of the parameters of the PSF model (e.g. PSF size),
and signal-to-noise-ratio of stars. We create two sets of simulations for which we can 
compute the CRLB, one with variable signal-to-noise ratio and constant FWHM, and one 
with variable FWHM and constant signal-to-noise ratio. After applying
different centroiding methods to the simulations, we investigate how close 
these methods can get to saturating the CRLB for various ranges of background 
Gaussian noise level and PSF FWHM.

In this work, we focus on four centroiding methods. The first method is the maximum-likelihood 
estimator which involves fitting a PSF profile, assuming that we have a good PSF estimate, to the star. 
The second method estimates the centroid of a star by fitting a 2d second-order polynomial to 
the 3$\times$3 patch around the brightest pixel of the image after convolution with the PSF. 
The third method centroids stars by
 smoothing the image of stars by a Gaussian kernel of a fixed size,
 and then applying the same 3$\times$3 polynomial trick to the smooth
 image. This method is fast and does not require any knowledge of the 
PSF. The last method we consider, is a center-of-light centroiding 
(measurement of a first moment), applied to the 7$\times$7 patch around the brightest pixel of the image.

This paper is structured as follows. In Section \ref{sec:CRLB},
we discuss the Cram\'{e}r-Rao lower bound, and we derive
an analytic expression for the lower bound on centroiding error
of the simulated data. 
In Section \ref{sec:method} we give a brief overview of 
centroiding methods used in our investigation.
In Section \ref{sec:data} we discuss the Cram\'{e}r-Rao lower bound satuaration
tests and their corresponding simulated data.
In Section \ref{sec:result}, we compare the performances of the methods
discussed in \ref{sec:method}, with the CRLB derived in \ref{sec:CRLB}. Finally,
 we discuss and conclude in Section \ref{sec:discussion}.               

%%%%%%%%%%%%%%%%%%%%Cram\'{e}r-Rao lower Bound %%%%%%%%%%%%%%

\section{Cram\'{e}r-Rao lower bound}\label{sec:CRLB}

Cram\'{e}r-Rao lower bound sets a limit, in some sense, on how well a measurement 
can be made in noisy data.  The bound can only be computed in the context of a 
generative model, or a probabilistic forward model of the data. That is, we can 
only compute the CRLB in the context of assumptions about the properties of the data. 
However, it makes sense for us to use centroiding methods that saturate the CRLB under 
some reasonable assumptions, even if we find that those assumptions are slightly wrong 
in detail in real situations.

The closer an estimator is to saturating the CRLB, the more information about the quantity that we 
need to estimate is preserved. The closer the root-mean-squared-error of a given estimator is to the bound,  
the more optimal---in terms of preseving the information---the estimator is. In order to test how information-preserving 
various centroiding methods are, we need to compare their 
performances against each other at saturating the Cram\'{e}r-Rao lower bound 
on centroiding error. 

The Cram\'{e}r-Rao inequality \citep{cramer} sets a lower bound on the 
root-mean-squared error of unbiased estimators. The CRLB is given by the square-root of the inverse of 
the Fisher information matrix $\mathcal{F}$. Thus, in order to find the CRLB, it is sufficient to compute the Fisher matrix. 
This computation relies on a set of assumptions:

\begin{itemize}
  \item Known, constant model observable with known dependence on the model parameters. 
        In this work the model observables are the presumed known Moffat PSF profiles, 
        and the model parameters are the centroids. 
  \item Known, stationary noise process. In the context of centroiding stars, this is equivalent to 
        having background limited noise from sky background and CCD readout noise.
  \item Uncorrelated Gaussian noise with no outliers. 
\end{itemize}

\todo{HOGG HELP: can you read the following paragraph and see whether it makes sense or not?}

Note that in this study, we explicitly focus on sky-limited images. In the sky-limited images, the contribution 
to the Poisson pixel noise is largely domiated by the sky rather than the objects. The exceptions are the very bright 
sources, but we only consider relatively low signal-to-noise ratio faint stellar sources. After subtraction of the 
sky background level, distribution of the Poisson pixel noise can be approximated by a Gaussian distribution with 
zero mean. \todo{Also mention in the discussion: The assumption of uncorrelated Gaussian noise is only an approximation 
to a Poisson distribution which has broader tails than what we expect from a Gaussian!} 
\todo{HOGG HELP: can you write a few words about uncorrelated noise?}

Let us assume that there are $M$ observables $\mathbf{f} = (f_{1}, ... , f_{M})$, each
related to $B$ model parameters $\boldsymbol{\mathbf{\theta}} = (\theta_{1} , ... , \theta_{B})$ 
\beq
f_{m} = f_{m}(\theta_{1} , ... , \theta_{B}).
\label{genmodel}
\eeq

Assuming uncorrelated Gaussian error with variance $\sigma^{2}_{m}$ for each observable $f_{m}$, elements
of the $B\times B$ Fisher matrix $\mathcal{F}_{ij}$ are given by
\beq
\mathcal{F}_{ij} = \sum_{m=1}^{M}\frac{1}{\sigma_{m}^{2}}\frac{\partial f_{m}}{\partial \theta_{i}}\frac{\partial f_{m}}{\partial \theta_{j}}
\label{fisher}
\eeq

Let us assume that, for each parameter $\theta_{i}$, there exists a set of 
asymptotically unbiased estimators $\{\hat{\theta}_{i}\}$. The Cram\'{e}r-Rao inequality 
states that the root-mean-squared error of this set is greater than or equal to 
the $i$-th diagonal element of the inverse of the Fisher information matrix:
\beq
\text{RMSE}[\{\hat{\theta_{i}}\}] \geq \sqrt{[\mathcal{F}^{-1}]_{ii}},
\label{inequality}
\eeq
where the left hand side of the inequality is called the Cram\'{e}r-Rao bound on 
the root-mean-squared error of estimating the parameter $\theta_{i}$. Note that 
the bound is computed assuming that the model (\ref{genmodel}) generating the data 
is known, and that uncertainties are given by additive uncorrelated Gaussian noise.

Based on inequality (\ref{inequality}), \citet{cramer} defines efficiency of unbiased 
estimators as the ratio of the CRLB and the root-mean-squared-error such that the maximum efficiency 
achievable by an estimator is unity. The closer the RMSE to the CRLB, 
the more information about the parameter of interest is preserved, and thus the more efficient 
the estimator is. 

Let us consider the case of a maximum likelihood estimate $\boldsymbol{\mathbf{\theta}}_{\text{ML}}$, 
where the likelihood function corresponds to the same generative assumptions that we used to compute the CRLB.

\begin{eqnarray}
\boldsymbol{\mathbf{\theta}}_{\text{ML}} &=& \argmax \mathcal{L}, \\
-2\ln \mathcal{L} &=& \sum_{m}\frac{1}{\sigma_{m}^{2}}( y_{m} - f_{m}(\boldsymbol{\mathbf{\theta}}))^{2}, \\
\end{eqnarray}
where $y_{m}$ is the $m$th component of the observed data $\mathbf{y}$
\beq
\mathbf{y} = \mathbf{f}(\boldsymbol{\mathbf{\theta}}_{\text{true}}) + \mathbf{n}.
\eeq

Asymptotically, maximum likelihood estimators can achieve maximum efficiency. That is, the RMSE from a 
large set of maximum likelihood estimates $\{\theta_{\text{ML}}\}$, 
becomes greater than or equal to the CRLB (see \citealt{cramer}; \citealt{lecam} for proof). 

However, the relation (\ref{inequality}) does not necessarily hold for biased estimators. That is, 
the root-mean-squared-error for a biased estimator can be smaller than the CRLB (see \citealt{lecam} for examples).
Therefore, we want to investigate the conditions under which the RMSE arising from a given centroiding method 
becomes close to the CRLB, or whether it can become equal to the CRLB in which case the method is \emph{saturating} 
the bound, or whether it can drop below the CRLB in which case the method is \emph{beating} the bound (and therefore the estimator must be biased).   

\todo{Given that the authors are using simulated data, a direct search for biases in the estimator outputs is far more powerful than using violations of the Cramer-Rao inequality to infer biased techniques.}
  
In this investigation, the model observables for the noisy data are the pixel-convolved PSF (PSF profile evaluated at different pixel locations), and  
the model parameters under consideration are the centroid coordinates. Therefore, $\mathcal{F}$
is a 2$\times$2 matrix whose elements are given by

\beq
  \mathcal{F}_{ij} = \sum_{m}\frac{1}{\sigma^{2}}
                \frac{\partial f_{m}}{\partial \theta_{i}}\frac{\partial f_{m}}{\partial \theta_{j}},
\label{fish}
\eeq
where the summation is over pixels, $f_{m}$ is the value of the PSF at pixel location $m$,
$\theta=\{x_{c},y_{c}\}$, and $\sigma^{2}$ is variance of the uncorrelated Gaussian noise map $n(\mathbf{x}_{m})$
\begin{eqnarray}
\mathbb{E}[n(\mathbf{x}_{m})] &=& 0, \\
\mathbb{E}[n(\mathbf{x}_{m})n(\mathbf{x}_{m^{\prime}})] &=& \sigma^{2}\delta_{m,m^{\prime}}. 
\end{eqnarray}

Derivation of an explicit expression for the Fisher matrix $\mathcal{F}$ requires 
specifying a presumed correct PSF model.
We use the Moffat profile \citep{moffat} for our PSF simulations. 
Moffat profile is an analytic model for stellar PSFs. It has broader wings than
a simple Gaussian profile. The surface brightness of the Moffat profile is given by
\beq
I(r) = \frac{F(\beta -1)}{\pi \alpha^{2}}[1+(r/\alpha)^{2}]^{-\beta},
\label{mof}
\eeq
where $F$ is the total flux, $\beta$ is a dimensionless parameter, and $\alpha$ is
the scale radius of the Moffat profile, with FWHM (hereafter denoted by $\gamma$)
being $2\alpha\sqrt{2^{1/\beta}-1}$. At a fixed $\gamma$, Moffat profiles with lower values
of $\beta$ have broader tails. It is also important to note that for sufficiently large values of the 
parameter $\beta$, the Moffat PSF becomes arbitrarily close to a simple Gaussian PSF. 

\todo{HOGG HELP: can you read the following paragraph and maybe add a few words to it?}

Note that in our PSF simulations, we assume that the individual generated images are 
Nyquist sampled. All pixels in the images are identical, and the stars are simulated 
by sampling from the pixel-convolved PSF. In well-sampled images, the center of 
the pixel-convolved PSF must be very close to the center of the optical PSF.
 
In order to investigate the performance of centroiding methods for
 different background noise levels and different
values of the parameter $\gamma$, simulation of a large number of images of stars---for which the exact positions of centroids
and their corresponding lower bounds are known---is required.

Given the PSF model (\ref{mof}), an expression for the CRLB as a function of the size, and SNR of stars can be 
derived. For further simplicity, the flux of all stars in our simulations are set to unity and per-pixel 
uncertainties are assumed to be uncorrelated Gaussian.

Moreover, it is more convenient to work with the signal-to-noise ratio
(hereafter denoted by SNR) instead of the variance of the Gaussian noise.
We use the definition of SNR according to which, SNR is given by the ratio
 of the mean and variance of the distribution
which the flux estimator is drawn from. Assuming that the total flux from
the point source is $F$, and that the sub-pixel shifted PSF at the $i$-th pixel is given
by $P_{i}$. Therefore the brightness of the $i$-th pixel $y_{i}$ is drawn from
a Gaussian distribution 
\beq
p(y_{i}) = \mathcal{N}(FP_{i},\sigma^{2}). 
\eeq

The optimal estimator of flux is the matched-filter flux estimator 
$\tilde{F}=\sum_{i}y_{i}P_{i}$. It can be shown that 
\beq
p(\tilde{F}) = \mathcal{N}(F , \frac{\sigma^{2}}{\sum_{i}P_{i}^{2}}),
\eeq  
which leads us to
\beq
\begin{array}{l}
\text{SNR} = \frac{F\sqrt{\sum_{i} P_{i}^{2}}}{\sigma}.
\end{array}
\label{snr}
\eeq

 In the case of Moffat profiles (\ref{mof}) with total flux of stars set to unity, 
SNR given in (\ref{snr}) can be analytically 
expressed in terms of the per pixel uncertainty
$\sigma$, FWHM $\gamma$, and also $\beta$
\beq
\text{SNR} = \frac{2(\beta-1)(2^{1/\beta}-1)^{1/2}}{\pi^{1/2}(2\beta-1)^{1/2}}\frac{1}{\sigma \gamma}.
\label{snr2}
\eeq

Equation (\ref{snr2}) implies that at a fixed $\gamma$ and background Gaussian noise 
with variance $\sigma^{2}$, stars with broader tails (lower $\beta$) have lower SNR.
On the other hand, stars with higher value of $\beta$ have higher SNR. 
For sufficiently large $\beta$---where the PSF can be
approximated by Gaussian profile---SNR is approximately given by $0.664/(\sigma\gamma)$.
Furthermore, at a fixed $\beta$ and variance of the background noise $\sigma^{2}$,
observed stars with higher $\gamma$ have lower SNR.  

Throughout this investigation, $\beta$ is held fixed at the fiducial value of $\beta=$ 2.5, where SNR
is given by the following expression
\beq
\text{SNR} \simeq \frac{0.478}{\sigma \gamma}\;\;\; \mathrm{for}\;\;\; \beta = 2.5.
\eeq

Given the analytic expression for the Moffat PSF model (\ref{mof}), and choice of $\beta=2.5$, 
the inverse of the Fisher matrix is given by
\beq
  \mathcal{F}^{-1} \simeq \Big(0.685 \frac{\gamma}{\text{SNR}}\Big)^{2} 
  \begin{pmatrix}
      1 & 0\\
      0 & 1\\
  \end{pmatrix}.
\label{crlbmoffat}
\eeq

Equation (\ref{crlbmoffat}) implies that at given SNR and $\gamma$,
CRLB for each component of centroid is approximately given by $0.685\gamma/\text{SNR}$,
and that a good centroiding technique delivers centroids with
root-mean-squared-error (hereafter RMSE) close to this. 

It is worth noting that for any PSF model whose radial light profile is some function of 
$r/\gamma$, CRLB has the same functional form, in that it is proportional to the ratio
between $\gamma$ and SNR. 
For PSF profiles with shorter tails (e.g., Gaussian), the prefactor of 0.685 in (\ref{crlbmoffat})
becomes smaller. In particular case of Gaussian PSF, the prefactor is approximately 0.6. 

\section{Centroiding methods}\label{sec:method}

In this section, we briefly discuss the approximate and the non-approximate centroiding methods considered in this 
study. 

\begin{description}

\item{{\bf Centroiding by fitting a correct PSF profile}} \quad We examine fitting an exact PSF 
profile to the stars. That is, 
in our Cram\'{e}r-Rao bound saturation tests, we find the best
estimates of flux and centroid by optimizing the likelihood using 
the correct PSF. We expect this method to perform best in determing the centroids
of stars, and deliver RMSE equal to Cram\'{e}r-Rao bound.

\item{{\bf Matched filter polynomial centroiding}} \quad Let us consider the case 
in which we have a good estimate of the pixel-convolved PSF at
the position of the faint star under consideration. 
We can smooth the image of the star, by correlating it with the 
full PSF $\mathcal{P}$ at the position of the star.
\begin{eqnarray}
Y^{(s)} &=& Y \star \mathcal{P}, \\
Y^{(s)}_{[i,j]} &=& \sum_{k,l}Y_{[i-k,j-l]}\mathcal{P}_{[k,l]},
\end{eqnarray}
where $Y$ is the image of the star, and $Y^{(s)}$ is sometimes called a matched filter. 
A matched filter is a method in which the data $Y$ is correlated (convolved in the 
case of symmetrical PSF) with the PSF $\mathcal{P}$. It is equivalent to optimizing the 
likelihood and therefore provides an optimal map 
where the peak of the map is the likely position of the 
point source (Lang \emph{et al.}, in preparation).

Then, we fit a simple 2d second-order polynomial 
$P(x,y)=a+bx+cy+dx^2+exy+fy^2$ 
to the 3$\times$3 patch centered on the brightest pixel of the
matched-filter image $Y^{s}$.
Upon constructing a universal 9$\times$6 design matrix
\begin{equation}
    \mathbf{A} = 
    \begin{bmatrix}
        1 & x_{1} & y_{1} & x_{1}^{2} & x_{1}y_{1} & y_{1}^{2} \\
        . & . & . & . & . & .  \\
        . & . & . & . & . & .  \\
        . & . & . & . & . & .  \\
        1 & x_{9} & y_{9} & x_{9}^{2} & x_{9}y_{9} & y_{9}^{2}
    \end{bmatrix},
\end{equation}
the free parameters $\{a,b,c,d,e,f\}$
(hereafter compactly denoted by $\mathbf{X}$) can be determined by 
\beq
\mathbf{X} = (\mathbf{A}^{T}\mathbf{A})^{-1}\mathbf{A}^{T}\mathbf{Z},
\label{linearfit}
\eeq
where $\mathbf{Z}$ is given by $(z_{1},...,z_{9})^{T}$,
with $z_{i}$, being the brightness of the $i-$th pixel of the 3$\times$3 patch centered on the brightest pixel of $Y^{(s)}$.
Afterwards, the best fit parameters can be used to compute the centroid coordinate

\beq
  \begin{bmatrix}
      x_{c}\\
      y_{c}\\
  \end{bmatrix} = 
  \begin{bmatrix}
      2d & e\\
      e & 2f\\
  \end{bmatrix}^{-1}
  \begin{bmatrix}
      -b\\
      -c\\
  \end{bmatrix}.
\label{center}
\eeq

It is important to note that the algebraic operation in (\ref{center}) involves 
inverting a 2$\times$2 curvature matrix
\beq
  D = 
  \begin{bmatrix}
      2d & e\\
      e & 2f\\
  \end{bmatrix}.
\eeq

When the curvature matrix $D$ has a zero (or very close to zero) deteminant,
centroid estimates obtained from equation (\ref{center}) can become arbitrarily 
large, which leads to catastrophic outliers. 
In order to tackle this issue, we add a soft regularization term
proportional to $\sigma$ to the diagonals of $D$ prior to inversion.

The procedure of convolving the image of star with the PSF results in a
smoother image. Therefore, a simple second-order polynomial will provide a better fit 
since convolution with the PSF makes the variation of the brightness of the image 
across the 3$\times$3 patch very smooth.

We should nonlinearly optimize a PSF to saturate the bound.
But because optimization is done through a chi-squared fitting, 
this is equivalent to optimizing a matched filter. 
And if the image is well sampled (in the PSF-convolved image) this
is equivalent to interpolating a matched filter on a grid. 
Therefore we expect the matched filter centroiding method to 
saturate the bound in cases where the image is well sampled.
 
\item{{\bf Default polynomial centroiding}} \quad In the case that 
we do not know the PSF at the position of star, we change 
the smoothing step in the following way. Instead of smoothing the image 
by convolving it with the PSF, smoothing is done by convolving the image 
with a default Gaussian kernel of a fixed size 
\beq
k(\mathbf{x}) = \frac{1}{2\pi w^2}\exp(-\mathbf{x}^{2}/2w^{2}),
\eeq
where throughout this study, the full-width at half-maximum of the Gaussian kernel is held at
a fixed value of 2.8 pixels (corresponding to w $\simeq$ 1.2 pixels). The smoothing step is done as follows
\begin{eqnarray}
Y^{(s)} &=& Y \star \mathcal{K}, \\
Y^{(s)}_{[i,j]} &=& \sum_{k,l}Y_{[i-k,j-l]}\mathcal{K}_{[k,l]},
\end{eqnarray}
where $Y$ is the image of the star, $Y^{(s)}$ is the smooth image, and $\mathcal{K}$ is a 7$\times$7 
array whose elements are given by the Gaussian kernel
\beq
\mathcal{K}_{[k,l]} = k(x_{k},y_{l}).
\eeq 
Then we apply the same 2d second-order polynomial method (see equations \ref{linearfit}, \ref{center}) to the 3$\times$3 patch centered on the brightest
pixel of the smooth image $Y^{(s)}$. Therefore, for a given star and a smoothing kernel,
the outcome of equation (\ref{linearfit}) can be
plugged into equation (\ref{center}) to find the centroid estimate
of the star. This is inspired by the 3$\times$3 quartic approximation 
used in the \textsl{Sloan Digital Sky Surveys} photometric pipeline \citep{sdss}.

\item{{\bf Center-of-light centroiding}} \quad In addition to the fitting methods 
mentioned so far, we examine centroiding stars by computing their first moments
in a 7$\times$7 patch around the brightest pixel of the image.

\todo{The use of a 7x7 kernel appears arbitrary. For the sake of a fair comparison with the matched filter, the size of the kernel should be equivalent to that used in the matched filter section.}

\begin{eqnarray}
x_{c} &=& \frac{\sum_{m}x_{m}Y_{m}}{\sum_{m}Y_{m}}, \\
y_{c} &=& \frac{\sum_{m}y_{m}Y_{m}}{\sum_{m}Y_{m}},
\end{eqnarray}
where the summation is done over all the pixels of the 7$\times$7 patch, and $x_{m}$, 
$y_{m}$, and $Y_{m}$, are the $x$ coordinate, $y$ coordinate, and the brightness
of pixel $m$ respectively.

In terms of saturating the Cram\'{e}r-Rao lower bound, we expect this simple 
center of light centroiding to perform worse than all other methods mentioned in 
this section. Hereafter, we call this method $7\times7$ moment centroiding.

\end{description}

\section{Tests}\label{sec:data}

We perform two sets of simulations. In the first set, we choose four values of
2, 2.8, 4, and 5.6 pixels for $\gamma$. For each $\gamma$, we generate 100,000 
17 $\times$ 17 postage-stamps of Moffat profiles with centroids randomly drawn
within the central pixel of the 17 $\times$ 17 postage-stamps. Moreover, zero-mean 
uncorrelated Gaussian noise is added to each postage-stamp such that the simulated 
stars are uniformly distributed in log-SNR between SNR = 5 to SNR = 100.

In the second set, we generate 100,000 17$\times$17 postage-stamps
of Moffat profile, with values of $\gamma$ uniformly distributed 
between 2 and 6 pixels, and with centroids drawn randomly within 
the central pixel. We choose four values for SNR: 5, 10, 20, and 40. 
For each SNR, and for each postage-stamp with a given $\gamma$, 
zero-mean uncorrelated Gaussian noise, with standard deviation corresponding 
to SNR and $\gamma$ through equation (\ref{snr2}), is added to each postage-stamp.

In the first experiment, we study how the centroiding error behaves with changing
SNR, while $\gamma$ is held constant. In the second experiment, we study 
how the centroiding error behaves with changing $\gamma$ while SNR is held constant.

\section{Results}\label{sec:result}

\subsection{Experiment 1}
   
In this experiment, after finding the centroiding error for each method,
we compute the RMSE in bins of SNR in order to compare it to the CRLB. 
Results of the first experiment are shown in Figures~\ref{1},~\ref{2},
~\ref{3},~\ref{4}. All methods deliver results with RMSE larger 
for fainter stars.

As we expected, the RMSE from centroiding by fitting the exact PSF model (Figure~\ref{1})
lies on the CRLB except for simulations with SNR $\la$ 10 where the RMSE gets slightly pulled away
from the CRLB due to presence of a few outliers. Figure~\ref{2} demonstrates that even the matched filter polynomial 
centroiding is able deliver centroiding estimates as accurate as PSF profile fitting, with the exception
of simulated stars with $\gamma$ = 2 pixels. For stars with $\gamma$ = 2 pixels, although second-order polynomial
fitting gets close to saturating the CRLB, it fails to saturate the CRLB since the PSF is slightly undersampled. 
For simulated images with higher $\gamma$, convolving the
data with the PSF results in images that are well-sampled around the brightest pixel. This allows the 
polynomial centroiding to deliver highly accurate results that can saturate the CRLB for simulated
data with SNR $\ga$ 10.

The RMSE from the default polynomial centroiding (Figure~\ref{3}),
is very close to the CRLB except at the very small values of SNR (SNR $\la$ 10).
As we increase $\gamma$ from 2 pixels to 2.8 pixels, RMSE gets closer
to the CRLB. When we increase $\gamma$ to
4 and 5.6 pixels, RMSE gets farther from the CRLB. For stars with $\gamma$ = 2 pixels, 
the rate at which the RMSE from this method drops
eventually becomes smaller than the constant rate at which the CRLB
decreases with increaing SNR. The reason for this is that even after smoothing
the data with a Gaussian kernel, the image is still relatively undersampled, and not smooth enough
for a second-order polynomial fitting to provide highly accurate centroiding estimates.
For stars with $\gamma=2.8$ pixels, since the FWHM of the Gaussian kernel matches that 
of the PSF of underlying simulations, the smooth images are well-sampled and therefore, 
the method is able to deliver estimates extremely close to saturating the CRLB. As $\gamma$
gets higher for the simulations, the convolved images are not as well-sampled as those 
whose $\gamma$ matches the FWHM of the smoothing kernel and as a result, we loose
some information by fitting a second-order polynomial to a 3$\times$3 patch. Therefore,
 we observe slight deviation from the CRLB for stars with $\gamma$ = 4 pixels, and 
slightly more deviation as we increase $\gamma$ to 5.6 pixels.

However, Figure~\ref{4} shows that in case of 7$\times$7 moment method, RMSE becomes
quite large as we move towards fainter stars in our simulation.
For stars with larger $\gamma$, centroid estimates from the naive center of light 
centroiding do not even come close to saturating the CRLB. As $\gamma$ increases, the 
RMSE deviates further from the CRLB.

\subsection{Experiment 2}

In this experiment, after finding the centroiding error for each method, we
compute the RMSE in bins of $\gamma$ in order to compare it to the CRLB. 
Behavior of error as a function of $\gamma$ for different values of SNR,
is shown in Figures~\ref{5},~\ref{6},~\ref{7}, and~\ref{8}. 
 
Once again, the RMSE from centroiding by fitting the exact PSF model as a function of FWHM
perfectly lies on the CRLB except at SNR = 5 where the RMSE slightly deviates
from the CRLB due to presence of a few outliers (see Figure~\ref{5}). 
Thus, centroid estimates from fitting
the exact PSF model always saturate the CRLB. Once again, we observe that the centroid
estimates found by matched filter polynomial centroiding saturate the CRLB with the
exception of simulated stars with SNR = 5, or $\gamma$ very close to 2 pixels (see Figure~\ref{6}).

Figure~\ref{7} illustrates that the default polynomial method results in RMSE very close to the CRLB.
For all four values of SNR, as we increase $\gamma$ from 2 pixels to 3 pixels,
RMSE gets slightly closer to the CRLB since the method starts to perform
slightly better as we move away from undersampled stars, and as the FWHM of the smoothing kernel
gets closer to that of the simulated images.
After approximately 3 pixels, increasing $\gamma$ results in deviation of RMSE of the method from the CRLB.
This is a characteristic of polynomial method as we apply it to a smooth image which is still not sufficiently
well-sampled.  Furthermore, increasing SNR from 5 to 40 makes the RMSE (as a function of $\gamma$) become
closer to the CRLB. In the case of extremely faint stars (SNR = 5),
default polynomial centroiding is not able to deliver any reliable estimate, and it fails.

The centroid estimates obtained from the naive 7$\times$7 moment method (see Figure ~\ref{8})
result in RMSE much larger than the CRLB in all ranges of FWHM and for all four
values of SNR in this experiment. 

\section{Discussion}\label{sec:discussion}

An efficient stellar centroiding algorithm must saturate---or come close to saturating---the fundamental 
Cram\'{e}r-Rao lower bound. That is, in all ranges of background noise level, size, radial light profile,
and shape, it must preserve information about the centroids of stars. In practice however,
this is only achievable when we have a reasonably good estimate of the PSF. Since we do not always 
know the exact PSF profile, we must make use of approximate centroiding algorithms. In this work, we
studied how close we get to saturating the CRLB with an approximate method acting on relatively low 
signal-to-noise ratio, unsaturated stars.
 
We focused on examples from two classes of centroiding algorithms. The first class contains fast and approximate
methods that do not require any knowledge of the PSF at the positions of stars. Of methods that belong to this class,
we consider centroiding stars based on fitting a second-order
polynomial to a 3$\times$3 patch of star images smoothed by a Gaussian kernel of fixed width, and finding the center of light 
of a 7$\times$7 patch around the brightest pixel of the star.
The second class of centroiding algorithms contains methods that require knowledge of the PSF (or having a good estimate
of the PSF) at the positions of stars. We considered two examples from this class. The first example
is the matched-filter polynomial centroiding, and the
second example is the full PSF profile fitting. In terms of saturating the Cram\'{e}r-Rao bound, we compared 
the performances of these methods against each other.

\todo{ The authors imply the PSF-fitting can only be used when the PSF is known. This is not true; one of the variables in PSF-fitting algorithms is often the PSF size (and sometimes shape) itself.}

Our results suggest that in all ranges of FWHM and SNR, the PSF fitting method returns 
centroid estimates that saturate the CRLB; with the only exceptions at SNR less than 10, 
in which case a few centroiding ouliers cause the RMSE to slightly deviate
from the CRLB. We found that the estimates found by 7$\times$7 moment method, except in the case of
very high SNR values and very small ranges of $\gamma$, do not come close to
saturating the CRLB. In a considerable range of PSF sizes and background noise levels, 
this method fails to deliver any reliable centroiding estimates.
 
\todo{The authors should quantify the difference in accuracy between the 7x7 moment method and an optimal estimator. Is it 10$\%$ worse? 10 times worse?}

On the other hand, the RMSE of centroid estimates of the default polynomial centroiding are much 
closer to saturating the CRLB in all ranges of signal-to-noise ratio even though this method does not require knowledge of 
the PSF.
The performance of this method however, is limited by
two important factors: (i) signal-to-noise ratio, and (ii) PSF size.
The default polynomial centroiding technique only takes advantage of the information 
contained in a 3$\times$3 patch centered on the brightest pixel of the smoothed image which is only 
well-sampled when the FWHM of the simulated image of star 
matches that of the smoothing kernel. Thus, when we apply 
this method to find the centroids of stars with larger FWHM, a certain amount of
information (encoded in the Cram\'{e}r-Rao lower bound) is lost, and therefore the RMSE
of these methods deviates from the CRLB. This deviation becomes larger at lower
signal-to-noise ratios. Besides, the performance of this method slightly degrades
in the case of undersampled stars (with FWHM close two 2 pixels). 
Presence of noise is another limitting factor.
Although this method is able to get very close to saturating the CRLB in a wide range of
signal-to-noise ratios, it is not reliable in the case of centroiding extremely faint stars. 
This is partly due to the fact that in the presence of noise, 
the brightest pixel of image does not necessarily contain the centroid of stars even after smoothing 
the image.

\todo{The intent of the statement that the default polynomial centroiding uses only a 3x3 pixel region of the smoothed image is unclear. While this is a factually true statement, the smoothing algorithm requires far more than 3x3 pixels on the original image to operate.}

Once we modify the polynomial method further by convolving the image with the full postage stamp
of the PSF, we obtain results that saturate the CRLB in a wide range of PSF sizes and noise levels.  
This is due to the fact that once the images of stars are convolved with the correct PSF, they become so well-sampled and smooth
that fitting a second-order polynomial to the 3$\times$3 patch centered on the brightest pixel of the smooth image is sufficient for us to obtain
results as accurate as those from fitting a PSF profile. The only conditions in which slight deviation from
saturating the CRLB occurs are very low signal-to-noise ratios (SNR $\la$ 10), and undersampled PSF (FWHM $\sim$ 2 pixels).
One advantage of this method over PSF fitting method is that it is fast, and that it is able to saturate the CRLB in 
a wide range of conditions. In the case that we have a good estimate of the PSF, the 
matched-filter polynomial centroiding is a significantly much faster algorithm. 
Although this method is more accurate than default polynomial centroiding, it is
slightly slower because it requires convolution of the image with the full potage stamp of the PSF. 

In this investigation we showed that PSF fitting always performs better---in terms of saturating the CRLB---at centroiding stars.
However, this method has its own disadvantages. First, we do not always know the the exact PSF. Second, finding 
the centroid by profile fitting is computationally expensive, whereas employing any of the 3$\times$3 polynomial techniques 
considered in this study in large scale astronomical surveys reduces the computational cost
of initial astrometry of the point sources considerably. 

\todo{If PSF-fitting is to be argued against for speed reasons, the authors should compare the speed of PSF-fitting astrometry and flux estimates against matched filter astrometry plus PSF-fitting flux estimation. It is not clear that the PSF-fitting only approach would be slower, given that the matched filter requires a convolution as well as something with the accuracy of PSF-fitting to estimate flux.}

Moreover, it is important to note that the PSF fitting method can be made faster by only keeping the 
term proportional to the dot product of the PSF model and the image in $\chi^2$. This is because 
the other terms in $\chi^{2}$ are not sensitive to the position of the centroid. However, this only allows
us to vary the poition of centroid, and not the flux, while fitting the PSF model
to the star. This is the advantage of a matched filter approach.

Modifying the $\chi^2$ such that it only contains the dot product of the PSF and image 
provides a nice interpretation. Finding a centroid coordinate that maximizes the dot product 
of the PSF and the star image is equivalent to finding the peak of the 
correlation of the PSF and the image. Therefore optimizing the modified $\chi^2$ is equivalent to finding the 
location of the peak of the matched filter. 
 
On the other hand, one can think of the initial smoothing step in the default polynomial method 
as upsampling of the image around the center by correlating it with an approximate Gaussian PSF. 
The default polynomial centroiding is different from PSF-fitting method in that, when there is mismatch between the
widths of the smoothing kernel and the that of the PSF, we loose some information by employing a 3$\times$3 polynomial fitting.
When we have the advantage of knowing the PSF, this issue can be resolved by employing the matched filter polynomial method.

Having a reasonable PSF model always helps us obtain more reliable centroid estimates, but
over a certain range of low signal-to-noise ratios and PSF sizes, one can achieve sensibly 
accurate results by employing a simple 3$\times$3 method after smoothing the image with 
a Gaussian kernel of a fixed width, and without making any assumption about the PSF model 
at the positions of stars.

\todo{It should be noted that estimation of a PSF within a few tenths of a pixel is trivial from most images. Thus, when dealing with realistic images, there is little if any difference between a true matched filter and the default polynomial.}

In this investigation we narrowed our focus on a set of data simulated from a particular
PSF profile. Although there are various cases where Moffat profiles provide reasonable
representations of the point spread function, these profiles are not generic enough to let us
reach a more general conclusion. Another important, and untapped, area of study
would be devising a model that infers the centroids of stars and point spread function
---in its full extent, not at the catalog level---across an astronomical image simultaneously.
This is beyond the scope of this study. 

This work was partially supported by the NSF (grants IIS-1124794 and AST-1517237), NASA (grant NNX12AI50G), 
and the Moore-Sloan Data Science Environment at NYU. We thank Jo Bovy and Dustin Lang for useful discussions.

\begin{thebibliography}{70}

\bibitem[Bertin \& Arnouts (1996)]{sextractor} Bertin, E., Arnouts, S., 1996,  A\&AS , 117, 393
\bibitem[Cram\'{e}r (1946)]{cramer} Cram\'{e}r, H., 1946, Mathematical methods of statistics, Princeton university press
\bibitem[Le Cam (1953)]{lecam} Le Cam, L. M., 1953, University of California publications in statistics, Vol. 1 (no. 11.), 277
\bibitem[Lobos \etal (2015)]{lobos} Lobos, R.~A., Silva, J.~F., Mendez, R.~A., Orchard, M., 2015, \pasp , 127, 1166
\bibitem[Lupton \etal (2001)]{sdss} Lupton, R. \etal, 2001,  arXiv:astro-ph/0101420
\bibitem[Schechter \etal (1993)]{dophot} Schechter, P. L., Mateo, M., Saha, A., 1993, \pasp , 105, 1342
\bibitem[Stetson (1987)]{daophot} Stetson, P. B., 1987, \pasp , 99, 191
\bibitem[Trujillo etal. (2001)]{moffat} Trujillo, I., \etal, 2001, \mnras , 328, 977

\end{thebibliography}

\clearpage

\begin{figure}[!htb]
\minipage{.8\textwidth}
  \includegraphics[width=\linewidth]{f1.eps}
\endminipage
\caption{Scatter plots showing the relation between error in centroid measurement
from fitting the exact PSF model to the stars and the signal-to-noise ratio of stars,
with FWHM of : 2 (upper left), 2.8 (upper right), 4 (lower left), and 5.6 (lower right)
pixels. In each scatter plot, the blue solid line represents the root-mean-squared-error, and the green dashed line represents CRLB.}\label{1}
\end{figure}

\begin{figure}[!htb]
\minipage{.8\textwidth}
  \includegraphics[width=\linewidth]{f2.eps}
\endminipage
\caption{Scatter plots showing the relation between error in centroid
measurement from the matched filter polynomial method and the signal-to-noise
ratio of stars, with FWHM of : 2 (upper left), 2.8 (upper right), 4 (lower left),
and 5.6 (lower right) pixels. In each scatter plot, the blue solid line represents the root-mean-squared-error, and the green dashed line represents CRLB.}\label{2}
\end{figure}

\begin{figure}[!htb]
\minipage{.8\textwidth}
  \includegraphics[width=\linewidth]{f3.eps}
\endminipage
\caption{Scatter plots showing the relation between error in centroid measurement from the default polynomial method and the signal-to-noise ratio of stars, with FWHM of : 2 (upper left), 2.8 (upper right), 4 (lower left), and 5.6 (lower right) pixels. In each scatter plot, the blue solid line represents the root-mean-squared-error, and the green dashed line represents CRLB.}\label{3}
\end{figure}

\begin{figure}[!htb]
\minipage{.8\textwidth}
  \includegraphics[width=\linewidth]{f4.eps}
\endminipage
\caption{Scatter plots showing the relation between error in centroid measurement from the 7$\times$7 moment method and the signal-to-noise ratio of stars, with FWHM of : 2 (upper left), 2.8 (upper right), 4 (lower left), and 5.6 (lower right) pixels. In each scatter plot, the blue solid line represents the root-mean-squared-error, and the green dashed line represents CRLB.}\label{4}
\end{figure}

%%%%%%%%%%%%%%%%%%%%%%%%%%%%%%%%%%%%%%%%%%%%%%%%%%%%%%%%%%%%%%%%%%%%%%%%%%% FWHM PLOTS %%%%%%%%%%%%%%%%%%%%%%%%%%%%%%%%%%%%%%%%%%

\begin{figure}[!htb]
\minipage{.8\textwidth}
  \includegraphics[width=\linewidth]{f5.eps}
\endminipage
\caption{Scatter plots showing the relation between error in centroid measurement
from fitting the exact PSF model and FWHM of stars, with SNR  of : 5 (upper left),
10 (upper right), 20 (lower left), and 40 (lower right). In each scatter plot,
the blue solid
 line represents the root-mean-squared-error, and the green dashed line represents CRLB.}\label{5}
\end{figure}

\begin{figure}[!htb]
\minipage{.8\textwidth}
  \includegraphics[width=\linewidth]{f6.eps}
\endminipage
\caption{Scatter plots showing the relation between error in centroid measurement
from the matched filter polynomial method and FWHM of stars, with SNR  of : 5 (upper left),
10 (upper right), 20 (lower left), and 40 (lower right). In each scatter plot, the blue solid
 line represents the root-mean-squared-error, and the green dashed line represents CRLB.}\label{6}
\end{figure}

\begin{figure}[!htb]
\minipage{.8\textwidth}
  \includegraphics[width=\linewidth]{f7.eps}
\endminipage
\caption{Scatter plots showing the relation between error in centroid measurement
from the default polynomial centroiding and FWHM of stars, with SNR  of 
: 5 (upper left), 10 (upper right), 20 (lower left), and 40 (lower right). In each
scatter plot, the blue solid
 line represents the root-mean-squared-error, and the green dashed line represents CRLB.}\label{7}
\end{figure}

\begin{figure}[!htb]
\minipage{.8\textwidth}
  \includegraphics[width=\linewidth]{f8.eps}
\endminipage
\caption{Scatter plots showing the relation between error in centroid measurement
from the 7$\times$7 moment method and FWHM of stars, with SNR of : 5 (upper left),
10 (upper right), 20 (lower left), and 40 (lower right). In each scatter plot, 
the blue solid
 line represents the root-mean-squared-error, and the green dashed line represents CRLB.}\label{8}
\end{figure}

\end{document}
